% LuaLaTeX文書; 文字コードはUTF-8
\documentclass[unicode,12pt]{beamer}% 'unicode'が必要
\usepackage{luatexja}% 日本語したい
\usepackage[ipaex]{luatexja-preset}% IPAexフォントしたい
\renewcommand{\kanjifamilydefault}{\gtdefault}% 既定をゴシック体に

\usepackage{amssymb,amsmath,ascmac}

\usepackage{multirow}
\usepackage{bm}

\graphicspath{{../fig/}}

\usepackage{tikz}
\usepackage{xparse}
\usetikzlibrary{shapes,arrows}
%% define fancy arrow. \tikzfancyarrow[<option>]{<text>}. ex: \tikzfancyarrow[fill=red!5]{hoge}
\tikzset{arrowstyle/.style n args={2}{inner ysep=0.1ex, inner xsep=0.5em, minimum height=2em, draw=#2, fill=black!20, font=\sffamily\bfseries, single arrow, single arrow head extend=0.4em, #1,}}
\NewDocumentCommand{\tikzfancyarrow}{O{fill=black!20} O{none}  m}{
\tikz[baseline=-0.5ex]\node [arrowstyle={#1}{#2}] {#3 \mathstrut};}

%目次スライド
\AtBeginSection[]{
  \frame{\tableofcontents[currentsection]}
}
%アペンディックスのページ番号除去
\newcommand{\backupbegin}{
\newcounter{framenumberappendix}
\setcounter{framenumberappendix}{\value{framenumber}}
}
\newcommand{\backupend}{
\addtocounter{framenumberappendix}{-\value{framenumber}}
\addtocounter{framenumber}{\value{framenumberappendix}} 
}

%%%%%%%%%%%  theme  %%%%%%%%%%%
\usetheme{Copenhagen}
% \usetheme{Metropolis}
% \usetheme{CambridgeUS}
% \usetheme{Berlin}

%%%%%%%%%%%  inner theme  %%%%%%%%%%%
% \useinnertheme{default}

% %%%%%%%%%%%  outer theme  %%%%%%%%%%%
\useoutertheme{default}
% \useoutertheme{infolines}

%%%%%%%%%%%  color theme  %%%%%%%%%%%
%\usecolortheme{structure}

%%%%%%%%%%%  font theme  %%%%%%%%%%%
\usefonttheme{professionalfonts}
%\usefonttheme{default}

%%%%%%%%%%%  degree of transparency  %%%%%%%%%%%
%\setbeamercovered{transparent=30}

% \setbeamertemplate{items}[default]

%%%%%%%%%%%  numbering  %%%%%%%%%%%
% \setbeamertemplate{numbered}
\setbeamertemplate{navigation symbols}{}
\setbeamertemplate{footline}[frame number]

\title{粘着・剥離の基礎と\\タッキファイヤーの働き}
\subtitle{~ 第一章 はじめに ~}
\author[東亞合成 佐々木]{佐々木 裕\thanks{hiroshi\_sasaki@mail.toagosei.co.jp}}
\institute[東亞合成]{東亞合成株式会社}
\date{2024/2/15}

\begin{document}

%%%%%
% 1 P
%%%%%
\maketitle

\begin{frame} 
    \tableofcontents[]
\end{frame} 


\section{自己紹介}
\subsection{自己紹介}
\begin{frame}
	\frametitle{自己紹介}
	\begin{itemize}
		\item 略歴
			\begin{itemize}
				\item 北海道大学で合成化学系の高分子化学を専攻
				\item 卒業後、東亞合成株式会社に入社し、現在に至る
			\end{itemize}
		\item  研究・開発歴
			\begin{itemize}
				\item 合成をベースとした光硬化型材料の研究開発に従事。
				\begin{itemize}
					\item 新規材料の開発において、各種の特性評価を実施
					\item その際に、レオロジー等の評価技術の重要性を痛感。
				\end{itemize}
				\item 個別の材料開発から材料評価技術の深掘りへ軸足を。
				\item 現在は、シミュレーションやレオロジーを主として\\研究活動を継続。
			\end{itemize}
		\item その経験からのモットー
			\begin{itemize}
				\item 「化学をベースに、尤もらしく」
				\item 「物理、数学、統計の考えを利用して」
				\item 「できるだけシンプルなモデルで。」
			\end{itemize}
	\end{itemize}
	
\end{frame}

\subsection{感じてきたこと}
\begin{frame}
	\frametitle{感じてきたこと}
		\begin{alertblock}{これまでの経験を通して感じてきたこと}
			\begin{itemize}
				\item 「新規なものを作り出す技術としての\alert{化学の有用性}」を何度も再認識してきました。
				\item 化学的なアプローチにおいては経験則を重視して、個別の理由を考えがち。
				\item 物理、数学、統計等の中にある「\alert{事象を客観視し、\\普遍性を大事}にする考え方」の重要性も痛感しました。
				\item もっとも役立ったのは、レオロジーから学んだ「\alert{マクロな応答をミクロな化学構造へと繋げる想像力}」でした。
				\item 数学や物理で用いられている「\alert{モデル化}」ということも非常に有用であると感じてきました。
			\end{itemize}
		\end{alertblock}
\end{frame}

\subsection{考え方のコツ}
\begin{frame}
	\frametitle{考え方のコツ}
		\begin{block}{感じてきたことをまとめ直すと、}
			\begin{itemize}
				\item 化学構造式と実際の物性の関係は非常に複雑。
				\begin{itemize}
					\item ややこしいものを、全部理解しようとしても無理。
					\item かと言って、単純化しすぎても役に立たない。
				\end{itemize}
				\item 「なぜそうなっているんだろう?」と考えてみる。
				\begin{itemize}
					\item 自分の言葉で理由を考えて、
					\item 人に説明できるように話の流れを作る。
					\item 流れの各ステップはできるだけ単純に。
				\end{itemize}
				\item できるだけシンプルな実験を
				\begin{itemize}
					\item 同時に仮定を複数設定しないこと。
					\item 実験前によく考えて計画を建てる。
					\item 一つずつ検証していく。
				\end{itemize}
			\end{itemize}
		\end{block}
\end{frame}

\section{本講座の進め方}

\subsection{理解へのアプローチ}
\begin{frame}
	\frametitle{理解へのアプローチ}
	\vspace{-2mm}
	\begin{block}{ざっくりと捕まえよう。}
		\begin{itemize}
			\item 個々の要素技術の基本を、イメージとして捉えて、
			\item 全体像をざっくりと捕まえれば、理解は一気に容易に。
		\end{itemize}
	\end{block}
	
	\pause
	\vspace{-2mm}
	\begin{alertblock}{粘着技術を簡単に言えば、}
		\begin{itemize}
			\item 粘着シートが短時間で貼り付き、
			\item その状態をそれなりの強さで維持して、
			\item 必要に応じて簡単に剥離できる。
		\end{itemize}
	\end{alertblock}

	\pause
	\vspace{-2mm}
	\begin{exampleblock}{タッキファイヤーの働きは?}
		\begin{itemize}
			\item タッキファイヤーとはどんなもので、
			\item どんな働きをしているのだろうか?
			\item なぜ、そんなふうな機能を有しているのだろうか?
		\end{itemize}
	\end{exampleblock}

\end{frame}


\subsection{見える化のすすめ}
\begin{frame}
	\frametitle{自分の中への落とし込み}
		\begin{block}{「何のためにやりたいのか?」を明確に}
			\begin{itemize}
				\item 目的がわからないと、ゴールが見えません。
				\item 仕事であれば、上司とよく相談しましょう。
				\item 自己啓発であれば、自分の本心をよく見極めましょう。
			\end{itemize}
		\end{block}
		\pause
		\begin{block}{「何をやりたいのか?」を常に意識}
			\begin{itemize}
				\item 因果関係をはっきりとつけましょう。
				\begin{itemize}
					\item 因 $\Leftarrow$ 原因
					\item 果 $\Leftarrow$ 結果
				\end{itemize}
				\item 図として捉えてみましょう。
				\begin{itemize}
					\item 複雑な実事象をできるだけ単純化して、
					\item 一目で理解できるようにしましょう。
				\end{itemize}
			\end{itemize}
		\end{block}
\end{frame}

\begin{frame}
	\frametitle{色々なモデル化}
	著者の場合:「さまざまな条件のもとで、幅広い検討対象に対してでも当てはめることのできるような汎用的なモデル」を考えることが役に立ってきました。
	\begin{exampleblock}{モデル化のすすめ}
		\begin{itemize}
			\item 適度な深さで尤もらしく
				\begin{itemize}
					\item 簡単すぎるものは例外が多い。
					\item 複雑化しすぎても過適応
						\begin{itemize}
							% \item n個のデータを、n次の関数でフィット
							\item 個々の現象にだけ適応可能
							\item モデル化する意味がない
						\end{itemize}
				\end{itemize}
			\item 欲しいもの
				\begin{itemize}
					\item 汎用的に使えるモデル
					\item 尤もらしく、実験事実を説明できるもの
				\end{itemize}
		\end{itemize}
	\end{exampleblock}
\end{frame}

\subsection{目指すもの}

\begin{frame}
	\frametitle{おすすめのやり方}
	
	\begin{center}
		{\Huge \textbf{「急がば回れ」}}
	\end{center}
	
	\vspace{5mm}
	\begin{alertblock}<2->{ざっくり全体像をイメージ}
		\begin{itemize}
			\item 慌てて結果を出そうとするのではなく、
			\begin{itemize}
				\item 心を落ち着けて、
				\item やるべきことを明確化してイメージ
			\end{itemize}
			\item イメージとして全体像をザックリと捕まえる
			\item 理解は一気に容易に
		\end{itemize}
	\end{alertblock}
\end{frame}

\begin{frame}
	\frametitle{イメージを大事に}
		実際の研究開発に役に立つように粘着関連技術を理解していただくために、以下のような点に気を付けて、説明していきたいと考えています。
		\begin{block}{ポイント}
			\begin{itemize}	
				\item イメージしやすい、直感的な理解を目指す。
					\begin{itemize}
						\item 全体を俯瞰した概念的な説明を。
						\item 多様な切り口からの説明を。
					\end{itemize}
				\item 大事なことは何度か繰り返す。
					\begin{itemize}
						\item 一度ではわかりにくいかも。
						\item 似たような内容を、ちょっと違う言葉で。
					\end{itemize}
				\item ゆっくり議論
					\begin{itemize}
						\item わかりにくいことは遠慮なく質問を。
						\item やりたいことを伝えてください。
					\end{itemize}
			\end{itemize}
		\end{block}
\end{frame}

\end{document}